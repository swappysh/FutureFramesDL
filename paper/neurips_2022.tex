\documentclass{article}


% if you need to pass options to natbib, use, e.g.:
    % \PassOptionsToPackage{numbers, compress}{natbib}
% before loading neurips_2022


% ready for submission
% \usepackage{neurips_2022}


% to compile a preprint version, e.g., for submission to arXiv, add add the
% [preprint] option:
\usepackage[preprint]{neurips_2022}


% to compile a camera-ready version, add the [final] option, e.g.:
% \usepackage[final]{neurips_2022}


% to avoid loading the natbib package, add option nonatbib:
% \usepackage[nonatbib]{neurips_2022}


\usepackage[utf8]{inputenc} % allow utf-8 input
\usepackage[T1]{fontenc}    % use 8-bit T1 fonts
\usepackage{hyperref}       % hyperlinks
\usepackage{url}            % simple URL typesetting
\usepackage{booktabs}       % professional-quality tables
\usepackage{amsfonts}       % blackboard math symbols
\usepackage{nicefrac}       % compact symbols for 1/2, etc.
\usepackage{microtype}      % microtypography
\usepackage{xcolor}         % colors


\title{Video Frame Prediction}


% The \author macro works with any number of authors. There are two commands
% used to separate the names and addresses of multiple authors: \And and \AND.
%
% Using \And between authors leaves it to LaTeX to determine where to break the
% lines. Using \AND forces a line break at that point. So, if LaTeX puts 3 of 4
% authors names on the first line, and the last on the second line, try using
% \AND instead of \And before the third author name.


\author{%
  Swapnil Sharma\thanks{\url{https://swappysh.github.io}} \\
  \texttt{swapnil.sh@nyu.edu} \\
  % examples of more authors
  \And
  Nikita Anand \\
  \texttt{email} \\
  \And
  Rishika Pabba \\
  \texttt{email} \\
  % \AND
  % Coauthor \\
  % Affiliation \\
  % Address \\
  % \texttt{email} \\
  % \And
  % Coauthor \\
  % Affiliation \\
  % Address \\
  % \texttt{email} \\
  % \And
  % Coauthor \\
  % Affiliation \\
  % Address \\
  % \texttt{email} \\
}


\begin{document}


\maketitle


\begin{abstract}
  Attempting Video frame prediction task. Things tried:
  \begin{enumerate}
    \item Semantic Segmentation Mask using ConvLSTM
    \item Tried Segformer model for sematic segmentation mask
    \item Using AutoEncoderDecoder for Semantic segmentation mask
    \item Training convLSTM for video frame prediction in auto regressive
      manner
    \item Tried convLSTM for video frame prediction in non auto regressive
      manner by predicting all frames at once
    \item Tried convBiLSTM for video frame prediction in non auto regressive
      manner by predicting all frames at once
    \item Implemented DataParallel for multiGPU training
    \item Normalized images before feeding to convLSTM
    \item Added skip connections in AutoEncoderDecoder for semantic segmentation
      mask
    \item Using ResNet AutoEncoderDecoder for semantic segmentation mask
    \item Using VPTR non auto regressive model for video frame prediction
  \end{enumerate}

  Best results for semantic segmentation mask were obtained using ResNet AutoEncoders. 
  Able to predict video frames using VPTR in auto regressive manner.
\end{abstract}

\section{Introduction}
This document describes the methodology and result analysis for the final Project 
in Deep Leanring Course. The task is to predict the semantic segmentation mask of 
22\textsuperscript{nd} frame given initial 11 frames of video. 

These videos have simple 3D shapes that interact with each other according to basic 
physics principles. Objects in videos have three shapes (cube, sphere, and cylinder), 
two materials (metal and rubber), and eight colors (gray, red, blue, green, brown, cyan, 
purple, and yellow). In each video, there is no identical objects, such that each 
combination of the three attributes uniquely identifies one object.

We tried breaking down the problem in semantic segmentation mask translation and 
future frame prediction tasks. To solve the first problem we tried using ConvLSTM, 
Segformer, and ResNet AutoEncoders. We got the best performance with ResNet AutoEncoders.

For the second task we tried using ConvLSTM, convBiLSTM and non autoregressive VPTR. Although 
we were not able to converge VPTR model on the entire dataset we got best performance on a subset 
with VPTR model.

\section{Related Work}
Most state-of-the-art (SOTA) models for video frame prediction use ConvLSTM-based AutoEncoders. 
These models were initially developed for predicting precipitation nowcasting, as introduced by 
\citet{Shi2015ConvolutionalLN}. They have been utilized in various studies, including 
\citet{Finn2016UnsupervisedLF}, \citet{Lotter2016DeepPC}, \citet{Xu2016EndtoEndLO}, 
\citet{Ballas2015DelvingDI}. According to \citet{Jing2019SelfSupervisedVF}, these models also 
work with self-supervised tasks.

Although the ConvLSTM-based models are flexible and efficient, they are generally slow due to 
recurrent prediction. To address this issue, standard CNNs or 3D CNNs and VAE-based methods 
have been proposed, such as those by \citet{Mathieu2015DeepMV} and 
\citet{Babaeizadeh2017StochasticVV}.

State-of-the-art models commonly rely on complex ConvLSTM models integrating attention mechanisms 
or memory-augmented modules. For instance, the Long-term Motion Context Memory model by 
\citet{Lee2021VideoPR} stores long-term motion context through a novel memory alignment learning, 
and the motion information is recalled during the test to facilitate long-term prediction. 
\citet{Chang2021MAUAM} proposed an attention-based motion-aware unit to increase the temporal 
receptive field of RNNs.

Almost all the state-of-the-art (SOTA) VFFP models are based on ConvLSTMs, i.e. convolutional 
short-term memory networks,which are efficient and powerful. Nevertheless as per \citet{Ye2022VPTRET}, 
they suffer from some inherent problems of recurrent neural networks(RNNs), such as slow training 
and inference speed, error accumulation during inference, gradient vanishing, and predicted 
frames quality degradation. Researchers keep improving the performance by developing more and 
more sophisticated ConvLSTM-based models. 

With the introduction of transformers, they have also been applied in the Vision domain, 
including video frame prediction. The ConvTransformer model by \citet{Liu2020ConvTransformerAC} 
follows the architecture of DETR introduced in \citet{Meinhardt2021TrackFormerMT}, a classical 
neural machine translation (NMT) Transformer architecture. DETR also inspired the development 
of the VPTR-NAR model by \citet{Ye2022VPTRET}, a non-autoregressive model for video frame 
prediction.


\section{Methodology}
We started with trying to predict the semantic segmentation mask of the frames given Normalized
images of the frames. We tried using ConvLSTM, Segformer and ResNet AutoEncoders

\section{Results}

\section{Submission of papers to NeurIPS 2022}


Please read the instructions below carefully and follow them faithfully.


\subsection{Style}


Papers to be submitted to NeurIPS 2022 must be prepared according to the
instructions presented here. Papers may only be up to {\bf nine} pages long,
including figures. Additional pages \emph{containing only acknowledgments and
references} are allowed. Papers that exceed the page limit will not be
reviewed, or in any other way considered for presentation at the conference.


The margins in 2022 are the same as those in 2007, which allow for $\sim$$15\%$
more words in the paper compared to earlier years.


Authors are required to use the NeurIPS \LaTeX{} style files obtainable at the
NeurIPS website as indicated below. Please make sure you use the current files
and not previous versions. Tweaking the style files may be grounds for
rejection.


\subsection{Retrieval of style files}


The style files for NeurIPS and other conference information are available on
the World Wide Web at
\begin{center}
  \url{http://www.neurips.cc/}
\end{center}
The file \verb+neurips_2022.pdf+ contains these instructions and illustrates the
various formatting requirements your NeurIPS paper must satisfy.


The only supported style file for NeurIPS 2022 is \verb+neurips_2022.sty+,
rewritten for \LaTeXe{}.  \textbf{Previous style files for \LaTeX{} 2.09,
  Microsoft Word, and RTF are no longer supported!}


The \LaTeX{} style file contains three optional arguments: \verb+final+, which
creates a camera-ready copy, \verb+preprint+, which creates a preprint for
submission to, e.g., arXiv, and \verb+nonatbib+, which will not load the
\verb+natbib+ package for you in case of package clash.


\paragraph{Preprint option}
If you wish to post a preprint of your work online, e.g., on arXiv, using the
NeurIPS style, please use the \verb+preprint+ option. This will create a
nonanonymized version of your work with the text ``Preprint. Work in progress.''
in the footer. This version may be distributed as you see fit. Please \textbf{do
  not} use the \verb+final+ option, which should \textbf{only} be used for
papers accepted to NeurIPS.


At submission time, please omit the \verb+final+ and \verb+preprint+
options. This will anonymize your submission and add line numbers to aid
review. Please do \emph{not} refer to these line numbers in your paper as they
will be removed during generation of camera-ready copies.


The file \verb+neurips_2022.tex+ may be used as a ``shell'' for writing your
paper. All you have to do is replace the author, title, abstract, and text of
the paper with your own.


The formatting instructions contained in these style files are summarized in
Sections \ref{gen_inst}, \ref{headings}, and \ref{others} below.


\section{General formatting instructions}
\label{gen_inst}


The text must be confined within a rectangle 5.5~inches (33~picas) wide and
9~inches (54~picas) long. The left margin is 1.5~inch (9~picas).  Use 10~point
type with a vertical spacing (leading) of 11~points.  Times New Roman is the
preferred typeface throughout, and will be selected for you by default.
Paragraphs are separated by \nicefrac{1}{2}~line space (5.5 points), with no
indentation.


The paper title should be 17~point, initial caps/lower case, bold, centered
between two horizontal rules. The top rule should be 4~points thick and the
bottom rule should be 1~point thick. Allow \nicefrac{1}{4}~inch space above and
below the title to rules. All pages should start at 1~inch (6~picas) from the
top of the page.


For the final version, authors' names are set in boldface, and each name is
centered above the corresponding address. The lead author's name is to be listed
first (left-most), and the co-authors' names (if different address) are set to
follow. If there is only one co-author, list both author and co-author side by
side.


Please pay special attention to the instructions in Section \ref{others}
regarding figures, tables, acknowledgments, and references.


\section{Headings: first level}
\label{headings}


All headings should be lower case (except for first word and proper nouns),
flush left, and bold.


First-level headings should be in 12-point type.


\subsection{Headings: second level}


Second-level headings should be in 10-point type.


\subsubsection{Headings: third level}


Third-level headings should be in 10-point type.


\paragraph{Paragraphs}


There is also a \verb+\paragraph+ command available, which sets the heading in
bold, flush left, and inline with the text, with the heading followed by 1\,em
of space.


\section{Citations, figures, tables, references}
\label{others}


These instructions apply to everyone.


\subsection{Citations within the text}


The \verb+natbib+ package will be loaded for you by default.  Citations may be
author/year or numeric, as long as you maintain internal consistency.  As to the
format of the references themselves, any style is acceptable as long as it is
used consistently.


The documentation for \verb+natbib+ may be found at
\begin{center}
  \url{http://mirrors.ctan.org/macros/latex/contrib/natbib/natnotes.pdf}
\end{center}
Of note is the command \verb+\citet+, which produces citations appropriate for
use in inline text.  For example,
\begin{verbatim}
   \citet{hasselmo} investigated\dots
\end{verbatim}
produces
\begin{quote}
  Hasselmo, et al.\ (1995) investigated\dots
\end{quote}


If you wish to load the \verb+natbib+ package with options, you may add the
following before loading the \verb+neurips_2022+ package:
\begin{verbatim}
   \PassOptionsToPackage{options}{natbib}
\end{verbatim}


If \verb+natbib+ clashes with another package you load, you can add the optional
argument \verb+nonatbib+ when loading the style file:
\begin{verbatim}
   \usepackage[nonatbib]{neurips_2022}
\end{verbatim}


As submission is double blind, refer to your own published work in the third
person. That is, use ``In the previous work of Jones et al.\ [4],'' not ``In our
previous work [4].'' If you cite your other papers that are not widely available
(e.g., a journal paper under review), use anonymous author names in the
citation, e.g., an author of the form ``A.\ Anonymous.''


\subsection{Footnotes}


Footnotes should be used sparingly.  If you do require a footnote, indicate
footnotes with a number\footnote{Sample of the first footnote.} in the
text. Place the footnotes at the bottom of the page on which they appear.
Precede the footnote with a horizontal rule of 2~inches (12~picas).


Note that footnotes are properly typeset \emph{after} punctuation
marks.\footnote{As in this example.}


\subsection{Figures}


\begin{figure}
  \centering
  \fbox{\rule[-.5cm]{0cm}{4cm} \rule[-.5cm]{4cm}{0cm}}
  \caption{Sample figure caption.}
\end{figure}


All artwork must be neat, clean, and legible. Lines should be dark enough for
purposes of reproduction. The figure number and caption always appear after the
figure. Place one line space before the figure caption and one line space after
the figure. The figure caption should be lower case (except for first word and
proper nouns); figures are numbered consecutively.


You may use color figures.  However, it is best for the figure captions and the
paper body to be legible if the paper is printed in either black/white or in
color.


\subsection{Tables}


All tables must be centered, neat, clean and legible.  The table number and
title always appear before the table.  See Table~\ref{sample-table}.


Place one line space before the table title, one line space after the
table title, and one line space after the table. The table title must
be lower case (except for first word and proper nouns); tables are
numbered consecutively.


Note that publication-quality tables \emph{do not contain vertical rules.} We
strongly suggest the use of the \verb+booktabs+ package, which allows for
typesetting high-quality, professional tables:
\begin{center}
  \url{https://www.ctan.org/pkg/booktabs}
\end{center}
This package was used to typeset Table~\ref{sample-table}.


\begin{table}
  \caption{Sample table title}
  \label{sample-table}
  \centering
  \begin{tabular}{lll}
    \toprule
    \multicolumn{2}{c}{Part}                   \\
    \cmidrule(r){1-2}
    Name     & Description     & Size ($\mu$m) \\
    \midrule
    Dendrite & Input terminal  & $\sim$100     \\
    Axon     & Output terminal & $\sim$10      \\
    Soma     & Cell body       & up to $10^6$  \\
    \bottomrule
  \end{tabular}
\end{table}

\bibliographystyle{plainnat}
\bibliography{references}

\end{document}